The CCSDS Subpacket Extractor can be used in a variety of different ways, each of which supports a different execution context.

The following diagram shows the most common way that the CCSDS Subpacket Extractor component might operate in the context of an assembly.

\begin{figure}[H]
  \includegraphics[width=0.9\textwidth,center]{assembly/build/eps/context_passive.eps}
  \caption{Example usage of a passive CCSDS Subpacket Extractor which extracts subpackets on the thread of its caller, and passes them along to a receiving component.}
\end{figure}

In the above context diagram the CCSDS Subpacket Extractor is made passive. This means that any CCSDS packet passed to it will be subpacket extracted on the thread of the calling component, since the calling component uses the \texttt{Ccsds\_Space\_Packet\_Recv\_Sync} connector.

Sometimes it is desireable to do the extracting on a different thread of execution to decouple this processing from work going on upstream of the component. The most obvious way to accomplish this is to make the CCSDS Subpacket Extractor component active, so it has its own thread of execution. The context diagram below shows this setup.

\begin{figure}[H]
  \includegraphics[width=0.9\textwidth,center]{assembly/build/eps/context_active.eps}
  \caption{Example usage of an active CCSDS Subpacket Extractor which extracts any subpackets from packets found on its queue using its own internal thread of execution.}
\end{figure}

In this case, the calling component passes CCSDS packets to the CCSDS Subpacket Extractor via the \texttt{Ccsds\_Space\_Packet\_Recv\_Async} connector, which puts the CCSDS packets on the CCSDS Subpacket Extractor's internal queue. When the CCSDS Subpacket Extractor is given execution time, it pops the CCSDS packets off of the queue and performs extraction.

It is also possible to achieve both synchronous and asynchronous routing using the same CCSDS Subpacket Extractor component. This might be useful if you have a normal data path which takes the asynchronous path, but a time critical data path which requires synchronous extraction. This might be the case when decoding time critical packets related to control or fault protection. The following context diagram shows this case.

\begin{figure}[H]
  \includegraphics[width=0.9\textwidth,center]{assembly/build/eps/context_mixed.eps}
  \caption{Example usage of a CCSDS Subpacket Extractor which extracts subpackets on the thread of its caller or its own thread depending on which connector is invoked.}
\end{figure}
